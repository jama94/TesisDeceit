\chapter*{\centering Resumen}
\addcontentsline{toc}{chapter}{Resumen}

Investigadores en psicología y en las ciencias de la computación están interesados en aplicar diferentes metodologías para detectar de manera automática y confiable si una persona dice declaraciones falsas o verdaderas.
Existen diferentes enfoques para detectar mentiras, tales como las reacciones fisiológicas, lenguaje verbal y lenguaje no verbal, que sugieren que una persona mentirosa tiene diferentes comportamientos a una persona que dice la verdad, pero estos comportamientos no son generalizables, ya que una pista que delata mentira en una persona no necesariamente indica mentira en otra. Esto ha sido el principal problema para desarrollar un sistema que sea automático y capaz de detectar mentiras en cualquier individuo. En este trabajo se presenta un modelo basado en redes neuronales profundas capaz de clasificar videos en verdades y mentiras, utilizando el lenguaje no verbal del rostro a partir de la aplicación de filtros espaciales a los fotogramas contenidos en un video, a bajo costo. sin ser un sistema invasivo y sin necesidad de expertos que deban analizar la información obtenida por el modelo manualmente. Los resultados obtenidos muestran que se pueden clasificar videos como verdad o mentira con el sistema calibrado con una exactitud del 82\% con la metodología de pruebas \textit{Within-Individual} y superando al juicio humano para detectar mentiras y verdades.\\
