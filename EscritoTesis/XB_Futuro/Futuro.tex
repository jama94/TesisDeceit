\chapter{Trabajo a futuro}
\label{TrabajoFuturo} 

Como trabajo a futuro, para mejorar el desempeño del modelo, está planeado incrementar el tamaño del dataset para el entrenamiento con datos completamente balanceado en vídeos, personas y tiempo, para así poder evaluar el modelo correctamente. Se tiene la hipótesis que al tener suficientes vídeos balanceados con gran cantidad de personas, la red pueda generalizar el problema, evitando la necesidad de calibrar el sistema para así obtener mejores resultados con nuevos individuos y convertirse en un modelo automático para detectar mentiras en vídeos. También para mejorar la exactitud del modelo, una nueva propuesta de preprocesamiento en la que propone aplicar los filtros espaciales antes del redimensionamiento y recorte, esto con el objetivo de perder resolución al momento de aplicar los filtros y así poder obtener filtros más precisos en bordes, movimiento de objetivos y en siluetas. \\

Adicionalmente este trabajo esta enfocado en clasificar vídeos como verdad o mentira, pero se propone que para trabajo a futuro que se desarrolle un sistema capaz de decidir tras las declaraciones hechas en diferentes vídeos, si una persona es confiable para un puesto de trabajo.\\