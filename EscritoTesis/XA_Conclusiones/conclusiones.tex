\chapter{Conclusiones}

Dentro del análisis expuesto se observa que existen varias pistas que pueden delatar si una persona esta diciendo una declaración falsa o verdadera; La mayoría de estas pistas pueden presentarse en la parte superior del cuerpo y es por esto que diferentes metodologías que se han propuesto, utilizan estas pistas para desarrollar diferentes sistemas que son capaces de detectar mentiras y verdades.
A pesar de que diferentes trabajos utilizan conjuntos de datos distintos para llevar a cabo las pruebas de su hipótesis, con este estudio se llegó a la conclusión de que el conjunto de datos que se utiliza debe consistir de ciertas características especiales para evaluar el modelo correctamente tales como que todos los individuos en el conjunto de datos deben aparecer mintiendo y diciendo verdades en múltiples ocasiones, en la misma cantidad de casos y con un escenario similar, para evitar tener una clasificación errónea causada por aspectos físicos. Las diferentes propuestas de modelos presentan varias problemáticas al momento de entrenar y evaluar el modelo profundo; Se propusieron soluciones a cada uno de las variantes de los modelos para resolver las problemáticas y así llegar al modelo final propuesto, el cuál tiene mayor exactitud para clasificar verdades y mentiras en vídeos.\\

Por último este estudio muestra un modelo que a través de la aplicación de filtros espaciales a los fotogramas de un vídeo y a través del uso de redes neuronales profunda, es capaz de clasificar vídeos como mentira o verdad cuando el modelo tiene vídeos previos de las personas a las cuáles se harán los experimentos, alcanzando el estado del arte de detección de mentiras por métodos no invasivos con las pruebas \textit{Within-Individual} y superando al humano en detectar mentiras.
El modelo tiene las ventajas de ser un sistema no invasivo ya que no requiere de instrumentos especializados de contacto ni de instrumentos costosos que requieran de metodologías que causen cambios fisiológicos debidos a consecuencias negativas y positivas; También tiene la ventaja de ser un modelo con ejecución automática ya que no requiere de personas que analicen la información obtenida por el sistema, solucionando el problema de tener una clasificación errónea causada por los prejuicios para decidir si un comportamiento delata mentira o verdad.